%$Id: thmtools.tex,v 1.3 2008/02/17 21:08:04 ulmi Exp ulmi $
\documentclass[a4paper, abstracton]{scrartcl}

\usepackage[T1]{fontenc}
\usepackage{fourier}
\usepackage[scaled]{helvet}

\addtokomafont{sectioning}{\fontseries{bc}\selectfont}

\newcommand\thmtools{\textsf{thmtools}}

\title{The \thmtools\ bundle\thanks{%
  This is a beta version. If you want to slightly ahead of your time,
  new development versions are availably from http://www.absatzen.de/thmtools.html
}}
\author{Ulrich M. Schwarz\thanks{ulmi@absatzen.de}}

\begin{document}
  \maketitle
  
  \begin{abstract}
    The \thmtools\ bundle provides several packages for commonly-needed 
    features for theorems. As designed, the bundle should work with kernel
    theorems, the theorem package and the amsthm package.
    
    \textbf{Warning:} Currently, things might still be a bit rough. You
    might want to consider not relying on \thmtools\ for your Ph.D. thesis
    masterpiece.
  \end{abstract}
  
  
  \section{thm-autoref}
  
  Fixes cooperation with hyperref's \verb|\autoref|. No further intervention
  on part of the user needed. (Loads aliasctr, thm-patch.)
  
  \section{thm-listof}
  
  Provides a \verb|\listoftheorems| command that works like
  \verb|\listoffigures|. Also provided: a \verb|\ignoretheorems| command that
  lets you exempt certain kinds:
  \begin{verbatim}
    \ignoretheorems{example,exercise,remark}
  \end{verbatim}
  There's currently no user interface to customize the appearance of the
  LoTheorems. (Loads thm-patch.)
  
  \section{thm-kv}
  
  Provides a key-value alternative to \verb|\newtheorem|, because I keep
  forgetting which optional argument goes where. Example:
  \begin{verbatim}
    \declaretheorem[unnumbered, title={Zorn's Lemma}]{zl}
    %\newtheorem*{zl}{Zorn's Lemma}
    \declaretheorem[numberwithin=section]{theorem}
    %\newtheorem{theorem}{Theorem}[section]
    \declaretheorem[sibling=theorem]{lemma}
    %\newtheorem{lemma}[theorem]{Lemma}
    \declaretheorem[numberlike=lemma]{axiom}
    %\newtheorem{axiom}[lemma]{Axiom}
  \end{verbatim}
  Supported keywords (several for the same functionality, because if you
  need to remember the keyword, you can just as well remember the order): 
  parent, numberwithin, within; sibling, numberlike, sharenumber;
  unnumbered, starred\footnote{works only with amsthm}; title, name,
  heading.
  
  Note: it's actually optional to give a title, in that case, the default is
  the name of the environment, with first letter uppercased.
  
  \section{thm-restate}
  
  Provides an environment \verb|restatable| that essentially puts the entire
  theorem into a single macro command for later re-statement:
  \begin{verbatim}
    \begin{restatable}[Well-ordering]{theorem}{wohlordnung}\label{thm:order}
      Every set is well-ordered.
   \end{restatable}
   And again: \wohlordnung
  \end{verbatim}
  Note that the label will be handled correctly (i.e. not re-defined).
  Limitation: verbatim-like things will not work. There's no handling for
  other counters, so it's not advisable to put floats inside a restatable.

  \section{aliasctr}
  
  Helper package, see separate doc.
  
  \section{unique}
  
  Helper package, see separate doc. Not used yet.
  
  \section{thm-patch}
  Helper package: redefines \verb|\newtheorem| so the defined theorems have
  hooks we can use. Note that the redefinition will always allow
  \verb|\newtheorem*| and giving contradicting optional arguments, but the
  backend original \verb|\newtheorem| will complain.

\end{document}
