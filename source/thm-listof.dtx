% \iffalse meta-comment
%
% Copyright (C) 2008 by Ulrich M. Schwarz
%
% This file may be distributed and/or modified under the conditions of
% the LaTeX Project Public License, version 1.3a.
% The license can be obtained from
% http://www.latex-project.org/lppl/lppl-1-3a.txt
%
% \fi
%
%\iffalse (hide this from DocInput)
%<*listof>
%\fi
%
% This package provides two main commands:
% \DescribeMacro{\listoftheorems}
% |\listoftheorems| will generate, well, a list of all theorems, lemmas,
% etc. in your document. This list is hyperlinked if you use \pkg{hyperref},
% and it will list the optional argument to the theorem.
%
% Currently, some options can be given as an optional argument keyval list:
% \begin{description}
%  \item[numwidth] The width allocated for the numbers, default 2.3em. Since
%  you are more likely to have by-section numbering than with figures, this
%  needs to be accessible.
%  \item[ignore=foo,bar] A last-second call to |\ignoretheorems|, see below.
%  \item[onlynamed=foo,bar] Only list those foo and bar environments that
%    had an optional title. This weeds out unimportant definitions, for
%    example. If no argument is given, this applies to all environments
%    defined by |\newtheorem| and |\declaretheorem|.
%  \item[show=foo,bar] Undo a previous |\ignoretheorems| and restore default
%    formatting for these environments. Useful in combination with
%    ignoreall.
%  \item[ignoreall]
%  \item[showall] Like applying ignore or show with a list of all theorems
%    you have defined.
% \end{description}
%
% \DescribeMacro{\listtheoremname}
% The heading name is stored in the macro |\listtheoremname| and is
% ``List of Theorems'' by default. All other formatting aspects are taken
% from |\listoffigures|. (As a matter of fact, |\listoffigures| is called
% internally.)
%
% \DescribeMacro{\ignoretheorems}
% |\ignoretheorems|\marg{remark,example,...} can be used to suppress some
% types of theorem from the LoTh. Be careful not to have spaces in the list,
% those are currently \emph{not} filtered out.
%
% There's currently no interface to change the look of the list. If you're
% daring, the code for the theorem type ``lemma'' is in |\l@lemma| and so on.
%
%\StopEventually{}
%    \begin{macrocode}
\let\@xa=\expandafter
\let\@nx=\noexpand
\RequirePackage{thm-patch,keyval,kvsetkeys}

\def\thmtlo@oldchapter{0}%
\newcommand\thmtlo@chaptervspacehack{}
\ifcsname c@chapter\endcsname
  \def\thmtlo@chaptervspacehack{%
    \ifnum \value{chapter}>\thmtlo@oldchapter\relax
      % new chapter, add vspace to loe.
      \addtocontents{loe}{\protect\addvspace{10\p@}}%
      \xdef\thmtlo@oldchapter{\arabic{chapter}}%
    \fi
  }%
\fi

\providecommand\listtheoremname{List of Theorems}
\newcommand\listoftheorems[1][]{%
  %% much hacking here to pick up the definition from the class
  %% without oodles of conditionals.
  \bgroup
  \setlisttheoremstyle{#1}%
  \let\listfigurename\listtheoremname
  \def\contentsline##1{%
    \csname thmt@contentsline@##1\endcsname{##1}%
  }%
  \@for\thmt@envname:=\thmt@allenvs\do{%
  \@xa\protected@edef\csname l@\thmt@envname\endcsname{% CHECK: why p@edef?
    \@nx\@dottedtocline{1}{1.5em}{\@nx\thmt@listnumwidth}%
  }%
  }%
  \let\thref@starttoc\@starttoc
  \def\@starttoc##1{\thref@starttoc{loe}}%
  % new hack: to allow multiple calls, we defer the opening of the
  % loe file to AtEndDocument time. This is before the aux file is
  % read back again, that is early enough.
  % TODO: is it? crosscheck include/includeonly!
  \@fileswfalse
  \AtEndDocument{%
    \if@filesw
      \@ifundefined{tf@loe}{%
        \expandafter\newwrite\csname tf@loe\endcsname
        \immediate\openout \csname tf@loe\endcsname \jobname.loe\relax
      }{}%
    \fi
  }%
  %\expandafter
  \listoffigures
  \egroup
}

\newcommand\setlisttheoremstyle[1]{%
  \kvsetkeys{thmt-listof}{#1}%
}
\define@key{thmt-listof}{numwidth}{\def\thmt@listnumwidth{#1}}
\define@key{thmt-listof}{ignore}[\thmt@allenvs]{\ignoretheorems{#1}}
\define@key{thmt-listof}{onlynamed}[\thmt@allenvs]{\onlynamedtheorems{#1}}
\define@key{thmt-listof}{show}[\thmt@allenvs]{\showtheorems{#1}}
\define@key{thmt-listof}{ignoreall}[true]{\ignoretheorems{\thmt@allenvs}}
\define@key{thmt-listof}{showall}[true]{\showtheorems{\thmt@allenvs}}

\providecommand\thmt@listnumwidth{2.3em}

\providecommand\thmtformatoptarg[1]{ (#1)}

\newcommand\thmt@mklistcmd{%
  \@xa\protected@edef\csname l@\thmt@envname\endcsname{% CHECK: why p@edef?
    \@nx\@dottedtocline{1}{1.5em}{\@nx\thmt@listnumwidth}%
  }%
  \ifthmt@isstarred
    \@xa\def\csname ll@\thmt@envname\endcsname{%
      \protect\numberline{\protect\let\protect\autodot\protect\@empty}%
      \thmt@thmname
      \ifx\@empty\thmt@shortoptarg\else\protect\thmtformatoptarg{\thmt@shortoptarg}\fi
    }%
  \else
    \@xa\def\csname ll@\thmt@envname\endcsname{%
      \protect\numberline{\csname the\thmt@envname\endcsname}%
      \thmt@thmname
      \ifx\@empty\thmt@shortoptarg\else\protect\thmtformatoptarg{\thmt@shortoptarg}\fi
    }%
  \fi
  \@xa\gdef\csname thmt@contentsline@\thmt@envname\endcsname{%
    \thmt@contentslineShow% default:show
  }%
}
\def\thmt@allenvs{\@gobble}
\newcommand\thmt@recordenvname{%
  \edef\thmt@allenvs{\thmt@allenvs,\thmt@envname}%
}
\g@addto@macro\thmt@newtheorem@predefinition{%
  \thmt@mklistcmd
  \thmt@recordenvname
}

\addtotheorempostheadhook{%
  \thmtlo@chaptervspacehack
  \addcontentsline{loe}{\thmt@envname}{%
    \csname ll@\thmt@envname\endcsname
  }%
}

\newcommand\showtheorems[1]{%
  \@for\thmt@thm:=#1\do{%
    \typeout{showing \thmt@thm}%
    \@xa\let\csname thmt@contentsline@\thmt@thm\endcsname
      =\thmt@contentslineShow
  }%
}

\newcommand\ignoretheorems[1]{%
  \@for\thmt@thm:=#1\do{%
    \@xa\let\csname thmt@contentsline@\thmt@thm\endcsname
      =\thmt@contentslineIgnore
  }%
}
\newcommand\onlynamedtheorems[1]{%
  \@for\thmt@thm:=#1\do{%
    \global\@xa\let\csname thmt@contentsline@\thmt@thm\endcsname
      =\thmt@contentslineIfNamed
  }%
}

\AtBeginDocument{%
\@ifpackageloaded{hyperref}{%
  \let\thmt@hygobble\@gobble
}{%
  \let\thmt@hygobble\@empty
}
\let\thmt@contentsline\contentsline
}

\def\thmt@contentslineIgnore#1#2#3{%
  \thmt@hygobble
}
\def\thmt@contentslineShow{%
  \thmt@contentsline
}

\def\thmt@contentslineIfNamed#1#2#3{%
  \thmt@ifhasoptname #2\thmtformatoptarg\@nil{%
    \thmt@contentslineShow{#1}{#2}{#3}%
  }{%
    \thmt@contentslineIgnore{#1}{#2}{#3}%
    %\thmt@contentsline{#1}{#2}{#3}%
  }
}
  
\def\thmt@ifhasoptname #1\thmtformatoptarg#2\@nil{%
  \ifx\@nil#2\@nil
    \@xa\@secondoftwo
  \else
    \@xa\@firstoftwo
  \fi
}
%    \end{macrocode}
%\iffalse
%</listof>
%\fi
