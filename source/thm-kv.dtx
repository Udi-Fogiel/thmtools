% \iffalse meta-comment
%
% Copyright (C) 2008 by Ulrich M. Schwarz
%
% This file may be distributed and/or modified under the conditions of
% the LaTeX Project Public License, version 1.3a.
% The license can be obtained from
% http://www.latex-project.org/lppl/lppl-1-3a.txt
%
% \fi
%
%\iffalse (hide this from DocInput)
%<*kv>
%\fi
%
%\StopEventually{}
%    \begin{macrocode}

\let\@xa\expandafter
\let\@nx\noexpand
\RequirePackage{keyval,kvsetkeys,thm-patch}

\newif\if@thmt@firstkeyset

% many keys are evaluated twice, because we don't know 
% if they make sense before or after, or both.
\def\thmt@trytwice{%
  \if@thmt@firstkeyset
    \@xa\@firstoftwo
  \else
    \@xa\@secondoftwo
  \fi
}

\@for\keyname:=parent,numberwithin,within\do{%
\define@key{thmdef}{\keyname}{\thmt@trytwice{\thmt@setparent{#1}}{}}%
}

\@for\keyname:=sibling,numberlike,sharenumber\do{%
\define@key{thmdef}{\keyname}{\thmt@trytwice{\thmt@setsibling{#1}}{}}%
}

\@for\keyname:=title,name,heading\do{%
\define@key{thmdef}{\keyname}{\thmt@trytwice{\thmt@setthmname{#1}}{}}%
}

\@for\keyname:=unnumbered,starred\do{%
\define@key{thmdef}{\keyname}[]{\thmt@trytwice{\thmt@isnumberedfalse}{}}%
}

\def\thmt@YES{yes}
\def\thmt@NO{no}
\def\thmt@UNIQUE{unless unique}
\define@key{thmdef}{numbered}[\thmt@YES]{
  \def\thmt@tmp{#1}%
  \thmt@trytwice{%
    \ifx\thmt@tmp\thmt@YES
      \thmt@isnumberedtrue
    \else\ifx\thmt@tmp\thmt@NO
      \thmt@isnumberedfalse
    \else\ifx\thmt@tmp\thmt@UNIQUE
      \RequirePackage[unq]{unique}
      \ifuniq{\thmt@envname}{%
        \thmt@isnumberedfalse
      }{%
        \thmt@isnumberedtrue
      }%
    \else
      \PackageError{thmtools}{Unknown value `#1' to key numbered}{}%
    \fi\fi\fi
  }{% trytwice: after definition
    \ifx\thmt@tmp\thmt@UNIQUE
      \addtotheorempreheadhook[\thmt@envname]{\setuniqmark{\thmt@envname}}%
      \addtotheorempreheadhook[\thmt@envname]{\def\thmt@dummyctrautorefname{\thmt@thmname\@gobble}}
    \fi
  }%
}


\define@key{thmdef}{preheadhook}{\thmt@trytwice{}{\addtotheorempreheadhook[\thmt@envname]{#1}}}
\define@key{thmdef}{postheadhook}{\thmt@trytwice{}{\addtotheorempostheadhook[\thmt@envname]{#1}}}
\define@key{thmdef}{prefoothook}{\thmt@trytwice{}{\addtotheoremprefoothook[\thmt@envname]{#1}}}
\define@key{thmdef}{postfoothook}{\thmt@trytwice{}{\addtotheorempostfoothook[\thmt@envname]{#1}}}

\define@key{thmdef}{style}{\thmt@trytwice{\thmt@setstyle{#1}}{}}

% ugly hack: style needs to be evaluated first so its keys
% are not overridden by explicit other settings
\define@key{thmdef0}{style}{%
  \ifcsname thmt@style #1@defaultkeys\endcsname
    \thmt@toks{\kvsetkeys{thmdef}}%
    \@xa\@xa\@xa\the\@xa\@xa\@xa\thmt@toks\@xa\@xa\@xa{%
      \csname thmt@style #1@defaultkeys\endcsname}%
  \fi
}
\kv@set@family@handler{thmdef0}{}% ignore everything else.

% fallback definition.
% actually, only the kernel does not provide \theoremstyle.
% is this one worth having glue code for the theorem package?
\def\thmt@setstyle#1{%
  \PackageWarning{thm-kv}{%
    Your backend doesn't have a `\string\theoremstyle' command.
  }%
}

\ifcsname theoremstyle\endcsname 
  \let\thmt@originalthmstyle\theoremstyle
  \def\thmt@outerstyle{plain}
  \renewcommand\theoremstyle[1]{%
    \def\thmt@outerstyle{#1}%
    \thmt@originalthmstyle{#1}%
  }
  \def\thmt@setstyle#1{%
    \thmt@originalthmstyle{#1}%
  }
  \g@addto@macro\thmt@newtheorem@postdefinition{%
    \thmt@originalthmstyle{\thmt@outerstyle}%
  }
\fi

\newif\ifthmt@isnumbered
\newcommand\thmt@setparent[1]{%
  \def\thmt@parent{#1}%
}
\newcommand\thmt@setsibling{%
  \def\thmt@sibling
}
\newcommand\thmt@setthmname{%
  \def\thmt@thmname
}

\thmt@mkextendingkeyhandler{thmdef}{thmdef}{\string\declaretheorem\space key}

\newcommand\declaretheorem[2][]{%
  \let\thmt@theoremdefiner\thmt@original@newtheorem
  \def\thmt@envname{#2}%
  \thmt@setthmname{\MakeUppercase #2}%
  \thmt@setparent{}%
  \thmt@setsibling{}%
  \thmt@isnumberedtrue%
  \@thmt@firstkeysettrue%
  \kvsetkeys{thmdef0}{#1}%
  \kvsetkeys{thmdef}{#1}%
  \protected@edef\thmt@tmp{%
    \@nx\newtheorem
    \ifthmt@isnumbered\else *\fi
    {#2}%
    \ifx\thmt@sibling\@empty\else [\thmt@sibling]\fi
    {\thmt@thmname}%
    \ifx\thmt@parent\@empty\else [\thmt@parent]\fi
  }%\show\thmt@tmp
  \thmt@tmp
  % uniquely ugly kludge: some keys make only sense
  % afterwards.
  % and it gets kludgier: again, the default-inherited
  % keys need to have a go at it.
  \@thmt@firstkeysetfalse%
  \kvsetkeys{thmdef0}{#1}%
  \kvsetkeys{thmdef}{#1}%
}

\providecommand\thmt@quark{\thmt@quark}

% in-document keyval, i.e. \begin{theorem}[key=val,key=val]

\thmt@mkextendingkeyhandler{thmuse}{thmuse}{\thmt@envname\space optarg key}

\addtotheorempreheadhook{%
  \ifx\thmt@optarg\@empty\else
    \@xa\thmt@garbleoptarg\@xa{\thmt@optarg}\fi
}%
\providecommand\thmt@garbleoptarg[1]{%
  \thmt@splitopt#1=\thmt@quark
  \ifcsname KV@thmuse@\thmt@tmpkey\endcsname
    % looks like a keyval-style argument
    \PackageInfo{thmtools}{kv-style argument to `\thmt@envname'}
    %\typeout{dbg: new-style arg `#1'}%
    \let\thmt@newoptarg\@empty
    \kvsetkeys{thmuse}{#1}%
    \let\thmt@optarg\thmt@newoptarg
  %\else
  %  \typeout{dbg: old-style arg `#1'}%
  \fi
%  % optarg present, check if it's newstyle
%  % a newstyle kv args starts with =.
%    %\tracingall
%  \if\expandafter\noexpand\@car #1\@nil=
%    \let\thmt@newoptarg\@empty
%    \@xa\def\@xa\thmt@optarg\@xa{\@cdr #1\@nil}%
%    \def\thmt@tempa{\setkeys{thmt-optarg}}%
%    \@xa\thmt@tempa\@xa{\thmt@optarg}% expansion!
%    \let\thmt@optarg\thmt@newoptarg
%  \else
%    \typeout{(dbg: old-style arg `#1')}%
%  \fi
}
\def\thmt@splitopt#1=#2\thmt@quark{%
  \def\thmt@tmpkey{#1}%
  \ifx\thmt@tmpkey\@empty
    \def\thmt@tmpkey{\thmt@quark}%
  \fi
  \@onelevel@sanitize\thmt@tmpkey
}

\define@key{thmuse}{label}{%
  %\typeout{setting label: #1}%
  \addtotheorempostheadhook[local]{\label{#1}}%
}
\define@key{thmuse}{name}{%
  %\typeout{optarg: #1}%
  \def\thmt@newoptarg{#1}%
}

%    \end{macrocode}
%
% Defining new theorem styles; keys are in opt-arg
% even though not having any doesn't make much sense.
% It doesn't do anything exciting here, it's up to 
% the glue layer to provide keys.
%
%    \begin{macrocode}
\def\thmt@declaretheoremstyle@setup{}
\def\thmt@declaretheoremstyle#1{%
  \PackageWarning{thmtools}{Your backend doesn't allow styling theorems}{}
}
\newcommand\declaretheoremstyle[2][]{%
  \def\thmt@style{#2}%
  \@xa\def\csname thmt@style \thmt@style @defaultkeys\endcsname{}%
  \thmt@declaretheoremstyle@setup
  \kvsetkeys{thmstyle}{#1}%
  \thmt@declaretheoremstyle{#2}%
}

\kv@set@family@handler{thmstyle}{%
  \PackageInfo{thmtools}{%
    Key `#1' (with value `#2')\MessageBreak 
    is not a known style key.\MessageBreak
    Will pass this to every \string\declaretheorem\MessageBreak
    that uses `style=\thmt@style'%
  }%
  \ifx\kv@value\relax% no value given, don't pass on {}!
    \@xa\g@addto@macro\csname thmt@style \thmt@style @defaultkeys\endcsname{%
      #1,%
    }%
  \else
    \@xa\g@addto@macro\csname thmt@style \thmt@style @defaultkeys\endcsname{%
      #1={#2},%
    }%
  \fi
}
%    \end{macrocode}
%
%\iffalse
%</kv>
%\fi
