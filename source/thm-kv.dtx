% \iffalse meta-comment
%
% Copyright (C) 2008 by Ulrich M. Schwarz
%
% This file may be distributed and/or modified under the conditions of
% the LaTeX Project Public License, version 1.3a.
% The license can be obtained from
% http://www.latex-project.org/lppl/lppl-1-3a.txt
%
% \fi
%
%\iffalse (hide this from DocInput)
%<*driver>
\documentclass{ltxdoc}

\usepackage[T1]{fontenc}
\usepackage{fourier}
\usepackage[scaled=0.8]{helvet}
\usepackage[scaled]{luximono}

\usepackage{amsmath, amsthm}
\usepackage{thm-kv, thm-patch}
\newtheorem{lemma}{Lemma}
\declaretheorem[sibling=lemma, shaded, name={Rule of Thumb}]{ruleofthumb}
\usepackage{color}
\declaretheorem[shaded={bgcolor={rgb}{1,0,0},rulecolor={rgb}{0,1,0},rulewidth=2em,
   margin=1em, textwidth=5cm}]{eyesore}


\GetFileInfo{thm-kv.sty}
\providecommand\pkg{\textsf}
\EnableCrossrefs
\CodelineIndex
\RecordChanges
\begin{document}

 \newcommand\thmrestate{\textsf{thm-kv}}
 \title{The \thmrestate\ package\thanks{%
   This file documents version~\fileversion\ of~\filedate,
   RCS ${}$Id: thm-kv.dtx,v 1.8 2008/06/22 12:09:41 ulmi Exp ulmi ${}$.
 }}
 \author{Ulrich M. Schwarz\thanks{ulmi@absatzen.de}}

 \maketitle

\begin{abstract}
\end{abstract}
  \DocInput{thm-kv.dtx}
\end{document}
%</driver>
%<*kv>
%\fi
%
% \CharacterTable
% {Upper-case \A\B\C\D\E\F\G\H\I\J\K\L\M\N\O\P\Q\R\S\T\U\V\W\X\Y\Z
% Lower-case \a\b\c\d\e\f\g\h\i\j\k\l\m\n\o\p\q\r\s\t\u\v\w\x\y\z
% Digits \0\1\2\3\4\5\6\7\8\9
% Exclamation \! Double quote \" Hash (number) \#
% Dollar \$ Percent \% Ampersand \&
% Acute accent \' Left paren \( Right paren \)
% Asterisk \* Plus \+ Comma \,
% Minus \- Point \. Solidus \/
% Colon \: Semicolon \; Less than \<
% Equals \= Greater than \> Question mark \?
% Commercial at \@ Left bracket \[ Backslash \\
% Right bracket \] Circumflex \^ Underscore \_
% Grave accent \` Left brace \{ Vertical bar \|
% Right brace \} Tilde \~}
% \CheckSum{160}
%
% \DoNotIndex{\@for,\addtocounter,\arabic,\csname,\endcsname,\cup,\CurrentOption}
% \DoNotIndex{\{,\},\do,\define@key,\def,\DeclareOption,\else,\ensuremath,\expandafter}
% \DoNotIndex{\hspace,\fi,\rule,\ifcase,\ifx,\in,\InputIfFileExists,\leq,\let,\mathpalette}
% \DoNotIndex{\NeedsTeXFormat,\ldots,\ldotp,\newcommand,\newcounter,\or}
% \DoNotIndex{\PackageInfo,\PackageWarning,\parm,\ProcessOptions,\protected@edef}
% \DoNotIndex{\providecommand,\ProvidesPackage,\relax,\renewcommand,\RequirePackage}
% \DoNotIndex{\setcounter,\setkeys,\rlap,\setminus,\widthof,\mathrm}
%
%\section{Usage}
%
% \DescribeMacro{\declaretheorem}
%The macro
%|\declaretheorem|\oarg{key=val-opts}\marg{name} can be used to define a
%new theorem instead of |\newtheorem|. It is hoped that |\declaretheorem| is
%easier to use than |\newtheorem|'s tangle of mutually-exclusive optional
%arguments. The following is the list of keywords understood:
%\begin{description}
%  \item[parent, numberwithin, within]
%    These keys govern when the theorem counter is reset. For example, giving
%    parent=chapter gives you theorems numbered per chapter, so it's
%    equivalent to the second optional argument to |\newtheorem|. There are
%    three names so you'll remember at least one of them.
%  \item[sibling, numberlike, sharenumber]
%    These keys make the theorem share a common numbering with the given
%    theorem. This is just like giving the first optional argument.
%  \item[unnumbered, starred]
%    If your theorem package supports it, this will call |\newtheorem*|,
%    i.e. you'll get a theorem that is never numbered. Currently, only
%    amsmath offers this possibility.
%  \item[name, title, heading] |\newtheorem| takes \emph{two} options, the
%    name of the environment (like |lem|) and its title (|Lemma|).
%    |\declaretheorem| only requires the environment name, and the title
%    defaults to the environment name with the first letter uppercased. If
%    you name your environments |lemma|, |theorem|, etc., you don't need to
%    do anything else. Otherwise, you can always manually specify the title.
%  \item[(pre/post)(head/foot)hook] You can specify extra code that will be
%    executed whenever you use the environment. \textbf{Warning:} this needs
%    the \pkg{thm-patch} package, and you're responsible for loading it
%    yourself if you want these keys to work. This functionality might be
%    shifted over to \pkg{thm-patch} in future releases.
%  \item[style]
%    This will issue a |\theoremstyle{foo}| for you if you give |style=foo|.
%    Note that currently, no care is taken to prevent this from becoming the
%    default style for subsequent theorems.
%\end{description}
% 
% \subsection{Examples}
% In many cases, you'll just get by with
% \begin{verbatim}
%  \declaretheorem{lemma}   
% \end{verbatim}
% which creates the environment `lemma', which will be labeled `Lemma' and
% numbered consecutively throughout:
%\begin{lemma}
%  This is what it looks like.
%\end{lemma}
% If you have more environments, you
% might want
% \begin{verbatim}
%  \declaretheorem[sibling=lemma]{theorem} 
% \end{verbatim}
% which will make the theorems share the numbering with the lemmas. If you
% had wanted per-chapter numbering for everything, you would have said
%\begin{verbatim}
%  \declaretheorem[parent=chapter]{lemma}
%\end{verbatim}
% without need to change subsequent declarations. A very fancy declaration
% using the shadethm and thm-patch package would look like this:
%\begin{verbatim}
%  \declaretheorem[sibling=theorem, shaded, name={Rule of Thumb}]{ruleofthumb}
%\end{verbatim}
%
%\begin{ruleofthumb}
% If all else fails, read the manual. Usually all else fails because you
% didn't.   
% \end{ruleofthumb}
%
% You can customize the colors and border like this:
%\begin{verbatim}
%\declaretheorem[shaded={bgcolor={rgb}{1,0,0},rulecolor={rgb}{0,1,0},rulewidth=2em,
%   margin=1em, textwidth=5cm}]{eyesore}
%\end{verbatim}
%\begin{eyesore}
%  But doing that is strongly discouraged.
%\end{eyesore}
%
% There is also an interface to the thmbox package: you can use |thmbox=X|, 
% where X is one of the styles L, M, S as defined by that package.
% (Actually, the parameter you give here is just passed on as optional
% argument to |\newboxtheorem|, so the other parameters like headstyle are valid as well.)
% The
% redefinition of proof and example is suppressed unless you load the thmbox
% package manually before you load thmtools.
%
%\StopEventually{}
%\section{Implementation}
%    \begin{macrocode}
\NeedsTeXFormat{LaTeX2e}
\ProvidesPackage{thm-kv}[2009/07/30 v0.1beta11 thm-kv interface (ulmi)]
\let\@xa\expandafter
\let\@nx\noexpand
\RequirePackage{keyval}

\define@key{thmt}{parent}{\thmt@setparent{#1}}
\define@key{thmt}{numberwithin}{\thmt@setparent{#1}}
\define@key{thmt}{within}{\thmt@setparent{#1}}

\define@key{thmt}{sibling}{\thmt@setsibling{#1}}
\define@key{thmt}{numberlike}{\thmt@setsibling{#1}}
\define@key{thmt}{sharenumber}{\thmt@setsibling{#1}}

\define@key{thmt}{title}{\thmt@setthmname{#1}}
\define@key{thmt}{name}{\thmt@setthmname{#1}}
\define@key{thmt}{heading}{\thmt@setthmname{#1}}

\define@key{thmt}{unnumbered}[]{\thmt@isnumberedfalse}
\define@key{thmt}{starred}[]{\thmt@isnumberedfalse}

\define@key{thmt}{preheadhook}{\addtotheorempreheadhook[\thmt@envname]{#1}}
\define@key{thmt}{postheadhook}{\addtotheorempostheadhook[\thmt@envname]{#1}}
\define@key{thmt}{prefoothook}{\addtotheoremprefoothook[\thmt@envname]{#1}}
\define@key{thmt}{postfoothook}{\addtotheorempostfoothook[\thmt@envname]{#1}}

\define@key{thmt}{style}{\thmt@setstyle{#1}}

\providecommand\theoremstyle[1]{%
  \PackageWarning{thm-kv}{%
    Your backend doesn't have a `\string\theoremstyle' command.\MessageBreak
    Your style request `#1' was ignored
  }%
}
\let\thmt@setstyle\theoremstyle

\newif\ifthmt@isnumbered
\newcommand\thmt@setparent[1]{%
  \def\thmt@parent{#1}%
}
\newcommand\thmt@setsibling{%
  \def\thmt@sibling
}
\newcommand\thmt@setthmname{%
  \def\thmt@thmname
}


\newcommand\declaretheorem[2][]{%
  \let\thmt@theoremdefiner\thmt@original@newtheorem
  \def\thmt@envname{#2}%
  \thmt@setthmname{\MakeUppercase #2}%
  \thmt@setparent{}%
  \thmt@setsibling{}%
  \thmt@isnumberedtrue%
  \setkeys{thmt}{#1}%
  \protected@edef\thmt@tmp{%
    \@nx\newtheorem
    \ifthmt@isnumbered\else *\fi
    {#2}%
    \ifx\thmt@sibling\@empty\else [\thmt@sibling]\fi
    {\thmt@thmname}%
    \ifx\thmt@parent\@empty\else [\thmt@parent]\fi
  }%\show\thmt@tmp
  \thmt@tmp
  % uniquely ugly kludge: some keys make only sense
  % afterwards.
  \setkeys{thmt}{#1}%
}
%    \end{macrocode}
% \subsection{Package-specific extensions}
% This is code that requires additional packages. These might be mutually
% incompatible.
%
% \subsubsection{shadethm}
%    \begin{macrocode}
  \define@key{thmt}{shaded}[{}]{%
    \RequirePackage{shadethm}%
    \RequirePackage{thm-patch}%
    \addtotheorempreheadhook[\thmt@envname]{%
      \setlength\shadedtextwidth{\linewidth}%
      \setkeys{thmt@shade}{#1}\begin{shadebox}}%
    \addtotheorempostfoothook[\thmt@envname]{\end{shadebox}}%
  }
%   There are some parameters you could set the default for (try them as is,
% first).
%    (i) shadethmcolor  The shading color of the background.  See the
%      documentation for the color package, but with a `gray' model, I find .97
%      looks good out of my printer, while a darker shade like .92 is needed
%      to make it copy well.  (Black is  0, white is 1.)
%    (i*) shaderulecolor  The shading color of the border of the shaded box.
%      See (i).  If \shadeboxrule is set to 0pt then this won't print anyway.
%    (i**) shadeboxrule  The width of the border around the shading.  Set it to
%      0pt (not just 0) to make it disappear.
%    (i***) shadeboxsep  The length by which the shade box surrounds the text.
\define@key{thmt@shade}{textwidth}{\setlength\shadedtextwidth{#1}}
\define@key{thmt@shade}{bgcolor}{\definecolor{shadethmcolor}#1}
\define@key{thmt@shade}{rulecolor}{\definecolor{shaderulecolor}#1}
\define@key{thmt@shade}{rulewidth}{\setlength\shadeboxrule{#1}}
\define@key{thmt@shade}{margin}{\setlength\shadeboxsep{#1}}
%    \end{macrocode}
% \subsubsection{thmbox}
% Emmanuel Beffara's thmbox package lets you draw certain sorts of borders
% around the theorems. Note that I don't think it honours \string\theoremstyle.
%    \begin{macrocode}
  \define@key{thmt}{thmbox}[L]{%
    \let\oldproof=\proof
    \let\oldendproof=\endproof
    \let\oldexample=\example
    \let\oldendexample=\endexample
    \RequirePackage[nothm]{thmbox}
    \let\proof=\oldproof
    \let\endproof=\oldendproof
    \let\example=\oldexample
    \let\endexample=\oldendexample
    \def\thmt@theoremdefiner{\newboxtheorem[#1]}%
  }%
%    \end{macrocode}
% \subsubsection{beamer}
% Nothing yet.
%    \begin{macrocode}
%    \end{macrocode}
%\iffalse
%</kv>
%\fi
